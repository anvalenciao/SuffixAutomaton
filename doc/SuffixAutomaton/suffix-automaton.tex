\section{Autómata de sufijo}\label{SuffixAutomaton}
Un autómata de sufijo es una estructura de datos eficiente y compacta, también conocido como Directed Acyclic Word Graph (DAWG), es el \acrshort{AFD} mínimo, que reconoce el conjunto de sufijos de una \gls{string} $S  = s_1 s_2 s_3 \dots s_n $ \cite{wiki:Suffix_automaton}, es decir, se puede usar un autómata de sufijo para determinar si una \gls{string} $S$ es una \gls{substring} en tiempo lineal en su longitud $O(| S |)$ \cite{article:10.1016/j.tcs.2009.03.034}.

\begin{theorem}[Principal]
El tamaño de un autómata sufijo de una \gls{string} $S$ es $O(| S |)$. El autómata puede ser implementado en tiempo $O(| S | \times log \; card(A))$ y $O(|S|)$ espacio extra \cite{book:Crochemore1997}.
\end{theorem}

%\gls{prefix}
%\gls{suffix}

\subsection{Propiedades}
% 24
El autómata de sufijo contiene información sobre todas las \glspl{substring} de la \gls{string} $S$. Para construir un autómata de sufijo en tiempo lineal, es necesario comprender dos conceptos \textit{End Positions} y \textit{Suffix Links}.

\subsubsection{End Positions (endpos)}
Los estados del autómata no son \glspl{substring}, los estados representan clases de equivalencia. Cada \gls{substring} de una \gls{string} pertenece a una clase de equivalencia llamada \textit{endpos} \cite{youtube:Automatas_de_Sufijos}.

\begin{definition}
	endpos(t): El conjunto de todas las posiciones en la \gls{string} $S$ donde termina la \gls{substring} $t$ \cite{web:suffix-automata}.
\end{definition}

\begin{example}
	Sea $t_1=``bc"$ y $t_2=``abc"$ \glspl{substring} de $S = ``abcdbc"$, entonces $endpos(t_1) = \{3, 6\}$ y $endpos(t_2) = \{3\}$ \cite{web:suffix-automata}.
	Sea $t=``bra"$ una \gls{substring} de $S = ``abracadabra"$, entonces $endpos(t) = \{3, 10\}$.
	Nótese que $endpos(t)$ es un conjunto cuyos elementos son posiciones finales de $t$ en todas sus ocurrencias en $S$ \cite{youtube:Automatas_de_Sufijos}.
\end{example}

Dos \glspl{substring}, $t_1$ y $t_2$, son endpos-equivalentes sí y sólo si $endpos(t_1) = endpos(t_2)$. 

\begin{example}
	Sea $t_1=``abra", t_2=``bra", t_3=``ra", t_4=``a"$ \glspl{substring} de $S = ``abracadabra"$, entonces $endpos(t_1) = \{3, 10\}, endpos(t_2) = \{3, 10\}, endpos(t_3) = \{3, 10\}, endpos(t_4) = \{0, 3, 5, 7, 10\}$.
	\begin{center}
		\begin{tabular}{lllllllllll}
			\multicolumn{11}{c}{S}                     \\
			a & b & r & a & c & a & d & a & b & r & a  \\
			0 & 1 & 2 & 3 & 4 & 5 & 6 & 7 & 8 & 9 & 10
		\end{tabular}
	\end{center}
	
	Por lo tanto, $endpos(t_1), endpos(t_2), endpos(t_3)$ son endpos-equivalentes, lo que equivale a decir que una de las palabras $t_1, t_2, t_3 $ es un \gls{suffix} de la otra. Nótese que $endpos(t_{1\leq3})$ y $endpos(t_4)$ no lo son.
\end{example}

\subsubsection{Suffix Links (endpos)}

\begin{definition}
	link(Q): punta a la clase de equivalencia \textit{endpos} definida por "el sufijo más largo de w que no pertenece a Q" \cite{web:suffix-automata}.
\end{definition}

% https://en.wikipedia.org/wiki/Suffix_automaton
% https://es.qaz.wiki/wiki/Suffix_automaton
% https://codeforces.com/blog/entry/20861
% https://static.googleusercontent.com/media/research.google.com/en//pubs/archive/35395.pdf
% https://stackoverflow.com/questions/24411000/what-is-a-suffix-automaton
% https://akshay.jaggi.co/blog/suffix-automata/
% https://saisumit.wordpress.com/2016/01/26/suffix-automaton/
% https://discuss.codechef.com/t/plant-editorial/45402
% https://cp-algorithms.com/string/suffix-automaton.html
% https://onlinejudge.org/index.php?option=onlinejudge&page=show_problem&problem=660
% https://sites.google.com/site/indy256/algo/suffix_automaton
% http://kiritow.com:3000/Kiritow/OJ-Problems-Source/src/e11349a42b5ca8a65621ba9cff78315deb482c15/.ACM-Templates/Data-Structures/suffix-automaton/main.cpp
% https://github.com/sourabh2311/Competitive-Programming/blob/master/SPOJ_NSUBSTR.cpp
% https://github.com/sourabh2311/Competitive-Programming/blob/master/UVA_719.cpp
% https://www.youtube.com/watch?v=axlgMCq7yTg&ab_channel=HowardCheng
% https://www.udebug.com/UVa/719
% https://www.wjyyy.top/1732.html
% https://hal-upec-upem.archives-ouvertes.fr/hal-00620792/document