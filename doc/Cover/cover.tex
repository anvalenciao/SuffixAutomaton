\begin{frontmatter}
	\title{Suffix Automaton} 
	\author{Andrés Valencia Oliveros\thanksref{myGitHub}\thanksref{myEmail}}
	\address{Facultad de Ingeniería, Diseño e Innovación\\ 
		Institución Universitaria Politécnico Grancolombiano\\
		Bogotá, Colombia
	}
	\thanks[myGitHub]{GitHub: 
		\href{https://github.com/anvalenciao/SuffixAutomaton}{\texttt{anvalenciao}}
	}
	\thanks[myEmail]{Email: 
		\href{mailto:anvalenciao@poligran.edu.co}{
			\texttt{\normalshape anvalenciao@poligran.edu.co}
		}
	}

	\renewcommand{\abstractname}{\textbf{Resumen}}
	\begin{abstract}
		Un autómata de sufijo es una estructura de datos eficiente y compacta, para representar el índice completo de un conjunto de cadenas. Es el autómata determinista mínimo que reconoce el conjunto de sufijos o subcadenas de un conjunto de cadenas, y se puede utilizar para búsqueda de patrones en textos. Este documento presenta definiciones y terminología de cadenas y autómatas, con una aplicación práctica que permite comprender de forma eficaz la eficiencia del algoritmo.
	\end{abstract}

	\begin{keyword}
		autómata finito, autómata de sufijo, coincidencia de cadenas.
	\end{keyword}
\end{frontmatter}